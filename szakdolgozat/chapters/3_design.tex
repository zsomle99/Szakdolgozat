\Chapter{Tervezés}

A szakdolgozatom lényege a tesztelés bemutatása és az alkalmazott technikák szemléltetése így maga a program ami készül hozzá, kimondottan ilyen céllal lett tervezve is, emiatt inkább kutatás jellegű a dolgozat. Ebben a részben a tesztelés tervezését részletezem illetve egy bővebben beszélek a szisztematikus tesztelésnek a szerepéről.
\Section{A tesztelés megtervezése}
Mint a szoftverfejlesztés, maga a tesztelés is egy tervel indul el.
A tesztelési terv (röviden: tesztterv) összefoglalja a fejlesztés és az üzemeltetés során elvégzendő tesztelési tevékenységeket. A tervezést a következő tényezők tudják befolyásolni:
\begin{enumerate}
\item A szervezet tesztstratégiája
\item A tesztelés irányelvei
\item A fejlesztési ciklus
\item Az alkalmazott módszerek
\item A teszt tárgya
\item A teszt céljai
\item A teszttel járó kockázatok
\item A teszthez tartozó megkötések
\item A kritikusság
\item Az elérhető erőforrások
\item A tesztelhetőség
\end{enumerate}

Egy projekt, illetve a tesztterv minél későbbi fázisban van annál több információ áll rendelkezésünkre és annál részletesebb tervet tudunk készíteni. A teszttervezést egy dinamikus tevékenység, amit minden fázisban el kell végezni. A teszttevékenységekből kapott információk alapján már jobban bele lehet tekinteni a kockázatokba és esetleges módosításokat lehet végezni ez alapján a tervbe.\\
A tervezést dokumentálhatják a fő teszttervben és az egyes tesztszintekhez tartozó teszttervekben, mint például a rendszertesztelés, elfogadási tesztelés, vagy a különböző teszttípusokhoz tartozó teszttervekben, mint például a használhatósági tesztelés és a teljesítménytesztelés\cite{syllabus3}.\\
A teszttervezési tevékenységekhez tartozhatnak és dokumentálhatóak a tervben :
\begin{itemize}
\item A kockázatoknak, a tesztelés tárgyának és a céloknak a definiálása
\item Az általános megközelítés kifejtése
\item Egy nagyobb leíró rész, arról, hogy mit fogunk tesztelni, azt ki fogja tesztelni, milyen erőforrásokra van ehhez szükség és hogyan kell a tevékenységeket végrehajtani
\item A tesztfelügyelethez és az irányításához a metrikák kiválasztása
\item A költségvetés meghatározása
\item A dokumentáció struktúrájának, illetve részletességének a meghatározása
\item A teszttevékenységek koordinálása, és az életciklusba beépítése
\item A terv különböző pontjai (elemzés,tervezés,megvalósítás,végrehajtás) ütemezése adott időpontokra, vagy iterációkhoz kötve.
\end{itemize}

\section {A tesztstratégia és a tesztelési megközelítés}

Maga a tesztstratégia a tesztelési folyamatot mutatja be általános megközelítéssel, többnyire egy szervezet vagy egy termék szintjén.\\
A következők az általános típusai a tesztstratégiáknak:
\subsection{Analitikus} Az analitikus megközelítés olyan típusa a tesztstratégiáknak, ahol egy bizonyos tényezőt veszünk középpontba (például kockázat vagy esetleg követelmény) és ez a tényező kerül elemzésre.
Kockázat alapú tesztelésnél azt jelenti az analitikus megközelítés, hogy az elkészülő teszt eseteket kockázat alapján priorizálják és tervezik meg.


\subsection{Modell alapú} Ahogy a neve is mutatja, a teszteket néhány modell alapján készítjük el, amik általában a termék egy jellemzőjére épülnek (például üzleti folyamat, belső szerkezet).	Az állapot modellek, a megbízhatósági növekedés modellek és az üzleti folyamat modellek tudnak példát adni ezekre a modellekre.


\subsection{Módszeres} Ez a tesztstratégia a tesztek és tesztfeltételek néhány előre meghatározott halmazát használja szisztematikusan. Ilyen például a fontos minőségi jellemzők listája, vagy a vállalati szintű megjelenés, de a  gyakori hibatípusok csoportjai is ide sorolhatóak.

\subsection{Folyamat szerinti} Ez a fajta stratégia külső szabályok és szabványokon alapul. Ez azt jelenti, hogy a tesztek tervezése, elemzése és megvalósítása ezeken alapul. Ezek a szabványok vagy a szervezet által vagy a szervezetre lettek meghatározva.

\subsection{Irányított} Ezt a stratégiát nagyban befolyásolja a technológiai szakértőktől vagy az érdekelt felektől érkező útmutatás vagy utasítás. Ezek a személyek általában a teszt csapaton, de akár a szervezeten kívüli személyek is lehetnek.

\subsection{Regresszió-kerülő} Az ilyen típusú tesztstratégia hátterében az a vágy áll, hogy elkerüljük a meglévő képességek romlását. Ez a tesztstratégia magában foglalja a meglévő tesztverek (különösen a tesztesetek és tesztadatok) újra felhasználását, a regressziós tesztek széles körű automatizálását és a standard tesztkészleteket\cite{syllabus4}.

\subsection{Reaktív} Az előző stratégiák helyett, ez a típus a tesztelés előre eltervezése helyett, inkább a komponens vagy a rendszer lényegére és a teszt végrehajtása során bekövetkezett eseményekre reagál. A stratégia a nevét is a reagálás rész miatt kapta. Nagyon gyakran a felderítő tesztelést alkalmazzák a reaktív stratégiákban\cite{syllabus4}.

