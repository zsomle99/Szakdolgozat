\Chapter{Bevezetés}



Maga a tesztelés a szoftverfejlesztési folyamatnak egy nagyon fontos része. Mindenki tapasztalt már hasonlót, hogy egy program nem úgy működött, ahogy annak kellett volna. Ez a hibás működés legyen az bármennyire kis mértékű, nagy időbeli és pénzbeli veszteségeket okozhat egy cégnek, sőt akár fizikai sérüléssel is járhat a jelenlegi szituációtól függően. A szoftvertesztelés lényege, hogy minimálisra csökkentse a kockázatát annak, hogy egy szoftver működés közben meghibásodik, illetve ellehetetlenítse egy olyan szoftver kiadását, amiben ismert hiba van, miközben lehetőséget ad a szoftverminőség kiértékelésére.

Ha valaki a meghallja a szoftvertesztelés szót, akkor gyakran tesztek írására és futtatására gondol, pedig maga a szoftvertesztelés egy teljes folyamat számos lépéssel, aminek csak egy része a teszt esetek megírása és az eredmények ellenőrzése.

Hasonlóan tévhit, hogy a szoftvertesztelés, csak a felhasználók által megszabott követelmények verifikációjára fókuszál, ugyanis bármilyen meg szabott speciális követelménynek való megfelelést is tartalmaznia kell, illetve a validációt is, ami már a működési környezetben ellenőrzi az igényeket.

Bármely projekt esetén a tesztelés céljai az alábbiak lehetnek:
\begin {enumerate}
 \item A munkatermékek, mint például a követelmények, felhasználói történetek, műszaki tervek és kód hibamegelőzés céljából történő kiértékelése

\item Annak igazolása, hogy az összes meghatározott követelmény teljesült-e


\item Annak ellenőrzése, hogy a teszt tárgya teljes mértékben implementálásra került, továbbá annak, validálása, hogy a felhasználók, illetve más érintett felek elvárásainak megfelelően működik.

\item A teszt tárgyának minőségébe vetett bizalom kiépítése

\item A meghibásodások és hibák megtalálása és ezáltal a nem megfelelő szoftverminőség kockázati szintjének csökkentése

\item Megfelelő információ biztosítása az érintett feleknek, hogy ezáltal megalapozott döntést hozhassanak különösen a teszt tárgyának minőségi szintjére vonatkozóan

\item A szerződésben foglalt, jogi vagy szabályozott követelményeknek, szabványoknak való megfelelés biztosítása, és/vagy igazolni a teszt tárgyának ezen követelményeknek, szabványoknak való megfelelőségét
\end {enumerate}
A tesztelés céljai változhatnak a tesztelt komponens vagy rendszer, a tesztszint és a szoftverfejlesztési életciklusmodell kontextusától függően\cite[~11. oldal]{syllabus}.


A szoftvertesztelés számos fajtával rendelkezik. Attól függően, hogy mi tartozik a tesztelés hatókörébe, különbséget teszünk a tesztelési fajták között. A szakdolgozat célja ezeknek a fajtáknak a leírása és bemutatása tesztelési környezetben, illetve környezet nélkül, amennyiben arra van lehetőség.
