\Chapter{Tervezés}

A szakdolgozatom lényege a tesztelés bemutatása és az alkalmazott technikák szemléltetése, így maga a program ami készül hozzá, kimondottan ilyen céllal lett megtervezve, emiatt inkább kutatás jellegű a dolgozat. Ebben a részben a tesztelés tervezését részletezem illetve bővebben beszélek a szisztematikus tesztelésnek a szerepéről.

\Section{A tesztelés megtervezése}

Mint a szoftverfejlesztés, maga a tesztelés is egy tervvel indul el.
A tesztelési terv (röviden: tesztterv) összefoglalja a fejlesztés és az üzemeltetés során elvégzendő tesztelési tevékenységeket. A tervezést a következő tényezők tudják befolyásolni:
\begin{enumerate}
\item a szervezet tesztstratégiája,
\item a tesztelés irányelvei,
\item a fejlesztési ciklus,
\item az alkalmazott módszerek,
\item a teszt tárgya,
\item a teszt céljai,
\item a teszttel járó kockázatok,
\item a teszthez tartozó megkötések,
\item a kritikusság,
\item az elérhető erőforrások,
\item a tesztelhetőség.
\end{enumerate}

Egy projekt, illetve a tesztterv minél későbbi fázisban van annál több információ áll rendelkezésünkre, és annál részletesebb tervet tudunk készíteni. A teszttervezés egy dinamikus tevékenység, amit minden fázisban el kell végezni. A teszttevékenységekből kapott információk alapján már jobban bele lehet tekinteni a kockázatokba és esetleges módosításokat lehet végezni ez alapján a tervbe\cite{nelson2009accelerated}.

A tervezést dokumentálhatják a fő teszttervben, és az egyes tesztszintekhez tartozó teszttervekben, mint például a rendszertesztelés, elfogadási tesztelés, vagy a különböző teszttípusokhoz tartozó teszttervekben, mint például a használhatósági tesztelés és a teljesítménytesztelés \cite[~67. oldal]{syllabus}.

A teszttervezési tevékenységekhez tartozhatnak és dokumentálhatóak a tervben:
\begin{itemize}
\item a kockázatoknak, a tesztelés tárgyának és a céloknak a definiálása,
\item az általános megközelítés kifejtése,
\item egy nagyobb leíró rész, arról, hogy mit fogunk tesztelni, azt ki fogja tesztelni, milyen erőforrásokra van ehhez szükség, és hogyan kell a tevékenységeket végrehajtani,
\item a tesztfelügyelethez és az irányításához a metrikák kiválasztása,
\item a költségvetés meghatározása,
\item a dokumentáció struktúrájának, illetve részletességének a meghatározása,
\item a teszttevékenységek koordinálása, és az életciklusba beépítése,
\item a terv különböző pontjai (elemzés, tervezés, megvalósítás, végrehajtás) ütemezése adott időpontokra, vagy iterációkhoz kötve.
\end{itemize}

\Section{A tesztstratégia és a tesztelési megközelítés}

Maga a tesztstratégia a tesztelési folyamatot mutatja be általános megközelítéssel, többnyire egy szervezet vagy egy termék szintjén.
A következő szakaszokban a tesztstratégiák általános típusait láthatjuk.

\SubSection{Analitikus}

Az analitikus megközelítés olyan típusa a tesztstratégiáknak, ahol egy bizonyos tényezőt veszünk középpontba (például kockázat vagy esetleg követelmény), és ez a tényező kerül elemzésre.
Kockázat alapú tesztelésnél azt jelenti az analitikus megközelítés, hogy az elkészülő teszt eseteket kockázat alapján priorizálják és tervezik meg.

\SubSection{Modell alapú}

Ahogy a neve is mutatja, a teszteket néhány modell alapján készítjük el, amik általában a termék egy jellemzőjére épülnek (például üzleti folyamat, belső szerkezet).	Az állapot modellek, a megbízhatósági növekedés modellek és az üzleti folyamat modellek tudnak példát adni ezekre a modellekre.

\SubSection{Módszeres}

Ez a tesztstratégia a tesztek és tesztfeltételek néhány előre meghatározott halmazát használja szisztematikusan. Ilyen például a fontos minőségi jellemzők listája, vagy a vállalati szintű megjelenés, de a  gyakori hibatípusok csoportjai is ide sorolhatóak.

\SubSection{Folyamat szerinti}

Ez a fajta stratégia külső szabályok és szabványokon alapul. Ez azt jelenti, hogy a tesztek tervezése, elemzése és megvalósítása ezeken alapul. Ezek a szabványok vagy a szervezet által vagy a szervezetre lettek meghatározva.

\SubSection{Irányított}

Ezt a stratégiát nagyban befolyásolja a technológiai szakértőktől vagy az érdekelt felektől érkező útmutatás vagy utasítás. Ezek a személyek általában a teszt csapaton, de akár a szervezeten kívüli személyek is lehetnek.

\SubSection{Regresszió-kerülő}

Az ilyen típusú tesztstratégia hátterében az a vágy áll, hogy elkerüljük a meglévő képességek romlását. Ez a tesztstratégia magában foglalja a meglévő \textit{tesztverek} (\textit{testware}) (különösen a tesztesetek és tesztadatok) újra felhasználását, a regressziós tesztek széles körű automatizálását, és a szabványos tesztkészleteket \cite[~68. oldal]{syllabus}.

\SubSection{Reaktív}

Az előző stratégiák helyett, ez a típus a tesztelés előre eltervezése helyett, inkább a komponens vagy a rendszer lényegére, és a teszt végrehajtása során bekövetkezett eseményekre reagál. A stratégia a nevét is a reagálás rész miatt kapta. Nagyon gyakran a felderítő tesztelést alkalmazzák a reaktív stratégiákban \cite[~68. oldal]{syllabus}.
