\Chapter{Összefoglalás}


\Section{Értékelés}

A dolgozat célja a szoftver tesztelés bemutatása volt. Ez alatt értendő, hogy milyen tesztelési módszereket mikor érdemes használni és, hogy mik ezek a módszerek.

A szakdolgozatban számos módszerrel találkozhattunk, amiket szöveges leírás után gyakorlati példával is bemutattam, így egy átfogóbb képet kaptunk a módszerekről.
Készült egy adatbázist kezelő program a szakdolgozat mellé, aminek a célja a módszerek alkalmazása a gyakorlatban volt, így számos funkció ennek megfelelően készült el, ami lehetséges, hogy egy cégnek készült szoftverben teljesen máshogy lenne implementálva.

A tervezettnél a programban az adatbázissal összeköttetéssel foglalkozó funkciók jobban készültek el. Sikerült egy funkciót megírni, ami le tudja kezelni önmagában a különböző parancsokat, ahelyett, hogy parancsonként külön-külön funkcióra legyen szükség.

A tervezettnél rosszabbul a fa struktúra elkészítése sikerült. Szerettem volna, hogy adatbázis független legyen a program és bármilyen hozzácsatolt adatbázissal működjön, viszont végül a tesztelés szempontjából, jobbnak láttam, hogyha egy hozzá elkészített adatbázissal foglalkozik csak a program.

\Section{További tervek}

Mindenképpen szeretném tovább fejleszteni a programot. A tervek közé tartozna, hogy lehessen vele adatbázist kreálni és törölni, így teljesen adatbázis függetlenné tenni a programot, illetve több adatbázissal is szeretném, ha tudna egyszerre foglalkozni.

Ezen felül szeretnék a grafikus felületen szépíteni, illetve még plusz elemeket hozzáadni, így a jelenlegi 4 gombos rész helyett, egy teljes panelt hozzáadni a jobb széléhez, ahol ezek a gombok elérhetőek lennének és a felszabadult helyen, pedig az éppen megnyitott adatbázis elemeit lehetne látni.
