\pagestyle{empty}

\noindent \textbf{\Large A mellékelt CD tartalma}

\vskip 1cm

Ebben a fejezetben a szakdolgozathoz mellékelt CD-nek a tartalmáról és használatáról lesz szó.\\
A CD-n található 1 \texttt{szakdolgozat.jar} nevezetű fájl, ami a program egy egyedülálló futtatható verziója. Ahhoz, hogy ezt el tudjuk indítani a következőre van szükség:
\begin{itemize}
	\item A Java JDK 18.0.1-es verziója fel van telepítve,
	\item A \texttt{.../jdk-18.0.1/bin} és \texttt{.../jdk-18.0.1/lib} mappákba az \newline \texttt{mssql-jdbc\_auth-10.2.0.x64.dll} nevezetű fájlt be kell másolni,
	\item A \texttt{SzakdolgozatDB.xlsx} vagy a \texttt{Szakdolgozat.sql} fájlt be kell importálni a Microsoft SQL Server Management Studio-ba.
\end{itemize}
Ezek után a program gond nélkül működik. Ha másik adatbázis kezelő programot szeretnénk használni, akkor a futtatható verziót nem tudjuk használni, viszont helyette a \texttt{program/Szakdolgozat\_program} mappában található fejlesztő környezetben íródott verziót igen, csak át kell írni a JDBC csatlakozási paramétert arra amit szeretnénk használni.

Ezeken felül még található egy \texttt{LaTeX} mappa, amiben megtalálható a szakdolgozat LateX forráskódja és egy \texttt{dolgozat.pdf} fájl, amiben a már lefordított szakdolgozat található.

Tovább megtalálható egy \texttt{README} MarkDown file, amiben megtalálható ez az útmutató.

Ha nem a futtatható verziót szeretnék használni, hanem a fejlesztő környezetben íródottat akkor a következő a teendő:
\begin{itemize}
	\item Az adott fejlesztő környezetet telepítsük fel (például: Eclipse, VS Code),
	\item Importáljuk be a \texttt{program/Szakdolgozat\_program} mappát.
\end{itemize}

